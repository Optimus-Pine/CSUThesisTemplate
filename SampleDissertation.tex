% This is a sample template created in the spring of 2016 by Melody Alsaker, which hopefully corrects some of the issues with Leif's csuthesis class. 
% The changes made in the preamble of this document correct formatting "problems" that the Grad School complained about.
% My dissertation was accepted by the Grad School in March 2016 using this template along with Leif's document class. 


\documentclass[nosmallcaps]{csuthesis} % The nosmallcaps command gets rid of the smallcaps font in the titles, which looks nice but is non-standard according to Grad School. 
\usepackage{amsfonts}
\usepackage{amsmath}
\usepackage{amssymb}
\usepackage{amsthm}
\usepackage{float} % Allows for override of object placement by using placement specifier H
\usepackage{graphicx}
\usepackage{placeins} % So we can use \FloatBarrier command -- very important to keep figures from migrating into chapters where they don't belong!
\usepackage[all]{nowidow} % This helps prevent widow lines, which the Grad School doesn't like!
\usepackage{lipsum}

% These commands help prevent orphan and widow lines by setting their penalties to "infinite badness." The Grad School forbids orphan and widow lines!!
\widowpenalty10000
\clubpenalty10000

% These two commands are needed to prevent large, awkward white spaces between sections and around equations, 
% which the Grad School apparently hates with the passion of a thousand fiery suns. 
% So you'll get gaps at the bottom of the page instead. As long as these aren't too big, it is okay.
% If you get very large white space at the bottoms of pages, they may complain about that, and then
% you may be stuck rewording your paragraphs. 
\raggedbottom % Tells LaTex it's okay to not fill text all the way to the bottom of the page.
\allowdisplaybreaks % Allows large blocks of equations to be split up across pages.



% This makes the bibliography left-aligned, as per Graduate School requirements. LaTex default will get rejected!!!
\makeatletter
\renewcommand{\@biblabel}[1]{[#1]\hfill}
\makeatother

% This is a modification of the commands to make chapter headings from Leif's template.
% Leif's template makes the font size larger and in small caps for chapter headings. Grad School DID NOT approve!!
% This code block overrides the chapter heading definitions in the .cls file. 
\def\@makechapterhead#1{\global\topskip .625in\relax
\begingroup
\normalsize\normalfont\scshape\centering
\ifnum\c@secnumdepth>\m@ne
\leavevmode \hskip-\leftskip
\rlap{\vbox to\z@{\vss
\centerline{\normalsize\mdseries
\uppercase\@xp{\chaptername}\enspace\thechapter}
\vskip 3pc}}\hskip\leftskip\fi
\uppercase{ #1 \large \par \endgroup} % REMOVING THE \Large COMMAND FROM CHAPTER TITLES, ADDED THE \large COMMAND INSTEAD, ADDED \uppercase
\skip@34\p@ \advance\skip@-\normalbaselineskip
\vskip\skip@ }


\title{Conformations to Idiotic Graduate School Formatting Requirements}
\author{Lando Calrissian}
\departmentname{Department of Mathematics}
\advisor{Jack Sparrow}
\committee{M.C. Hammer \and Wavy Gravy  \and Tyrion Lannister}
\gradterm{Spring}




\setcounter{secnumdepth}{2} % Section numbering depth (This can be changed if desired)
\setcounter{tocdepth}{2} % Table of contents depth (This can be changed if desired)



%%%%%%%%%%%%%%%%%%%%%%%%%%%%%%%%%%%%%%%%%%%%%%%%%%%
%%%%%%%%%%%%%%%%%%%%%%%%%%%%%%%%%%%%%%%%%%%%%%%%%%%
%%%%%%%%%%%%%%%%%%%%%%%%%%%%%%%%%%%%%%%%%%%%%%%%%%%
\begin{document}
\frontmatter

\begin{abstract}
Abstract goes here.
\end{abstract}

\begin{acknowledgements}
So long, and thanks for all the fish. 
\end{acknowledgements}

\maketitle
\tableofcontents
\cleardoublepage
\listoftables
\listoffigures


%%%%%%%%%%%%%%%%%%%%%%%%%%%%%%%%%%%%%%%%%%%%%%%%%%%
%%%%%%%%%%%%%%%%%%%%%%%%%%%%%%%%%%%%%%%%%%%%%%%%%%%
%%%%%%%%%%%%%%%%%%%%%%%%%%%%%%%%%%%%%%%%%%%%%%%%%%%
\mainmatter
\chapter{Introduction}
\lipsum[1] 

See Figure~\ref{dog} for a lovely picture of a dog doing science. \lipsum[2]

\begin{figure}[htbp]
\includegraphics{science-dog.jpg}
\caption{Figure captions go under the figures.} \label{dog}
\end{figure}
%%%%%%%%%%%%%%%%%%%%%%%%%%%%%%%%%%%%%%%%%%%%%%%%%%%
%%%%%%%%%%%%%%%%%%%%%%%%%%%%%%%%%%%%%%%%%%%%%%%%%%%

\section{{This is the first section}} 
\lipsum[3] See, for example, \cite{Item1, Item2, Item3, Item4} or \cite{Item5, Item6, Item7, Item8, Item9, Item10, Item11}. \lipsum[4]



%%%%%%%%%%%%%%%%%%%%%%%%%%%%%%%%%%%%%%%%%%%%%%%%%%%
\subsection{This is a subsection}
See some made-up numbers that mean nothing in Table~\ref{table1}. \lipsum[4-5].

\begin{table}[htbp]
\caption{Table captions go above the tables.} \label{table1}
\begin{tabular}{|c| c|}
\hline
Column1 & Column2 \\
\hline
4.5 & 900 \\
36 & 10.456 \\
\hline
\end{tabular}
\end{table}

\lipsum[6-8]


\FloatBarrier % Keeps floats from migrating past the bibliography (or into other chapters)

% Bibliography should begin on its own page. 
\newpage
\bibliographystyle{plain}
\bibliography{Sample}

\end{document}













